\documentclass{metanorma}

%% Temporary fixes %%%%%%%%%%%%%%%%%%%%%%%%%%%%%%%%%%%%%%%%%%%%%%%%%%%%%%%%%%%%%
\newenvironment{aside}{}{}
\newenvironment{comment}{}{}
\let\cmt\relax% [comment]#...#
\let\bcp\relax% [bcp14]#...#
\usepackage{verbatim}
\newenvironment{sourcecode}{\verbatim}{\endverbatim}
\usepackage{longtable}
\usepackage{multirow}
%%%%%%%%%%%%%%%%%%%%%%%%%%%%%%%%%%%%%%%%%%%%%%%%%%%%%%%%%%%%%%%%%%%%%%%%%%%%%%%%

\title{The Holy Hand Grenade of Antioch}

\set{author}{Arthur son of Uther Pendragon}
\set{email}{arthur.pendragon@ribose.com}
\set{doctype}{internet-draft}
\set{abbrev}{Hand Grenade of Antioch}
\set{updates}{8140}
\set{submission-type}{independent}
\set{name}{draft-camelot-holy-grenade-01}
\set{status}{informational}
\set{consensus}{false}
\set{area}{General, Operations and Management}
\set{keyword}{rabbits, grenades, antioch, camelot}
\set{ipr}{trust200902}
\set{toc-include}{true}
\set{sort-refs}{true}
\set{revdate}{2018-04-15T00:00:00Z}
\set{fullname}{Arthur son of Uther Pendragon}
\set{forename_initials}{A.}
\set{lastname}{Pendragon}
\set{organization}{Camelot}
\set{uri}{http://camelot.gov.example}
\set{street}{Palace Camel Lot 1}
\set{city}{Camelot}
\set{region}{England}
\set{country}{United Kingdom}
\set{comments}{yes}
\set{notedraftinprogress}{yes}
\set{smart-quotes}{false}
\set{docfile}{draft-camelot-holy-grenade.adoc}
\set{mn-document-class}{ietf}
\set{mn-output-extensions}{xmlrfc2,txt,html,nits}

\begin{document}

\maketitle

\tableofcontents

% [.comment]
% tag::preamble1[]
% //tag::preamble[]

\section{Abstract}

The menagerie of beasts and artefacts depicted in RFC8140
may be usefully supplemented by other renowned figures of
Internet and more general lore. This document extends the
menagerie to the seminal fable of the
"Holy Hand Grenade of Antioch", as depicted in the
Monty Python film "Monty Python and the Holy Grail",
as well as "Spamalot", the musical inspired by the movie.

\begin{note}
  \mn{remove-in-rfc=false}
  \caption{Spamalot}
  The relevance of the musical "Spamalot" to Internet lore should be
  obvious to the reader; but in case of doubt, see also
  Section 1 ("\textbf{What is Spam?}") of RFC2635.
\end{note}

% //end::preamble[]
% [.comment]
% end::preamble1[]

% [.comment]
% tag::sectnums1[]
% //tag::sectnums[]

% [toc=exclude]
% :sectnums!:
\subsection{Terminology}

The key words "\textbf{MUST}", "\textbf{MUST NOT}", "\textbf{REQUIRED}", "\textbf{SHALL}",
"\textbf{SHALL NOT}", "\textbf{SHOULD}", "\textbf{SHOULD NOT}", "\textbf{RECOMMENDED}",
"\textbf{NOT RECOMMENDED}", "\textbf{MAY}", and "\textbf{OPTIONAL}" in this document
are to be interpreted as described in BCP 14 \mncite{RFC2119} \mncite{RFC8174}
when, and only when, they appear in all capitals, as shown here.

:sectnums:
\subsection{Introduction}

\mncite{RFC8140} refers to the intended move of RFC formatting to
XML2RFC v3 \mncite{RFC7990}, in the following terms:

% //end::sectnums[]
% [.comment]
% end::sectnums1[]

% [.comment]
% tag::quote1[]
% //tag::quote[]

\begin{quote}
  \mn{attribution="A. Farrel"}
  Although the RFC Editor has recently dragged the IETF kicking and
  screaming into the twentieth century [RFC7990] [RFC7996], there is a
  yearning among all right-thinking Internet architects to "keep it
  simple" and to return to the olden days when pigs could be given
  thrust without anyone taking undue offence.
\end{quote}

% //end::quote[]
% [.comment]
% end::quote1[]

While no pigs, flying or otherwise, are involved in the transition
to RFC XML v3, it is opportune to enhance the \mncite{RFC8140}
legendarium in the service of RFC XML v3, by illustrating its
functionality through references to the mythology of Camelot, and
particularly the incidents at the Cave of Caerbannog.

% [.comment]
% tag::escaped_hyperlink1[]
% //tag::escaped_hyperlink[]

The screaming move into the twenty-\textbf{first} century is accompanied by
a move back to the late twentieth century, with ASCII stylings more
wonted in haunts like \href{ftp://ftp.wwa.com/pub/Scarecrow} (known to be
accessible in 1996.)

% //end::escaped_hyperlink[]
% [.comment]
% end::escaped_hyperlink1[]

There are two references to rabbits in
\textit{Monty Python and the Holy Grail} which are expounded on
herewith:

% [.comment]
% tag::listcontinuation1[]
% //tag::listcontinuation[]

\begin{description}
  \item[Trojan Rabbit]
    In their siege of the French-occupied castle which may already
    contain an instance of the Grail, Sir Bedevere the Wise proposes to
    use a Trojan Rabbit to infiltrate the castle, with a raiding party
    to take the French "not only by surprise, but totally unarmed."

    The proposal, unsurprisingly, proved abortive. The more so as the
    raiding party forgot to hide within the Trojan Rabbit, before the
    French soldiers took the Trojan Rabbit inside the castle.

  \item[Killer Rabbit of Caerbannog]
    Guarding the entrance to the Cave of Caerbannog; see \mncite{caerbannog}.

\end{description}

% //end::listcontinuation[]
% [.comment]
% end::listcontinuation1[]

\subsection{The French-occupied castle}

% [.comment]
% tag::inline_formatting1[]
% //tag::inline_formatting[]

The participants of that renowned exercise in cross-cultural
communication, to wit the exchange between the
\textit{Knights of the Round Table}
and the taunting French soldiers serving under 
\textbf{Guy de Lombard} are,
% TODO: properly speaking, outside the scope of this `menagerie`, being more
% TODO: or less human. Notwithstanding, several^ish^ beasts both animate~d~
and wooden played a significant part in this encounter; most
notably:

\begin{itemize}
  \item The Projectile Cow, see \mncite{projectile-cow}
  \item The Trojan Rabbit, see \mncite{trojan-rabbit}
\end{itemize}

% //end::inline_formatting[]
% [.comment]
% end::inline_formatting1[]


\begin{figure}[h]\centering
  \mn{alt=The Projectile Cow with an accompanying cannon in ASCII}
  \label{projectile-cow}
  \caption{The Projectile Cow with an accompanying cannon}
\begin{verbatim}
.-.-.-.-.-.-.-.-.-.-.-.--.-.-.-.-.-.-.--.-.-.-.-.-.-.-.--.-.
_-_---__--__--___-___-__-____---___-________---____-____-__-
._.-.-.-.-.-.-.-.-.-.-.-.-.-.-.-.-.-.-.--..-.-.-.-.-.-..--.-
,..,.,.,.,.,..,.,,..,.,.,.,.,.,,  ^^  .,,.,.,  ^^   .,.,.,.=
_>-.-.-.-._>_>_>_.-.-.-.-.-.-.-.  \\\  .,.,.  ///  .-.-.-.-.
.,.,.,.,..,.,..,.,.,..,.,.,,..,.,  \ \_______/ /    .,.,.,.,
.,.,.,.,..,.,.,.,..,,..,,.,.,.,.,.  <[ {o} . ]>  #   .,.,.,.
.-.-.--.-.-.-.-.-.--.-.-.-.--.-.-.   [ ______]       .-.-.-.
.-.--.-.-.-.--.-.-.-.--.-.-.,.,.,  / [ !  ` `]   .,.,..,.,.-
.,.,.,.-.-,l,-,l.-,.,.,.,-.,*.    /  {_!MOO!_}    . ., . . ,
.-.-.-.-.-.-.-.-.-.-.-.-.-.-    /M      /    -.-<>.,.,..-.-,
.-.-.--.-.-.-.-.-.-.-.-.--..   /MI    LK\____    .-.-.-.-.-.
.-.-.-.--.-.-.-.-.-.-.-.-.-   /MILK   mil_____k   ,.,.,..-,-
.-,-.-,-.,-.-,-.`-.-/-..     //    -`  //       .-.p . .-.-.
.-.--.-.-.-.-.-.-.-.        //   .,   //    .-.-.-.-.-.-.-.-
.-.-.--.-.-.-.-.-.-.  %____============    .-.-.--.-.-.-.-.-
-.-.-.-.--.-.-.-.-.-.      !  !           .,-.-.-,-,--,-.-,-
,--.-.-,--.--.-.,--,        \ \      .-,-,--.-,--,-.---,-.-,
,-.-.-,-,-.-,-,-.--,         +  >    .-,--,-.--,-,-.-.-,--,-
,--.-,--,-,--.---,-               .-,-,--.--,--,-.---,-,-.-.
.,.,.,.,..,.,.,.{A\      .,.,.,.,..,.,.,.,.,.,..,.,.,.,..,.,
.,.,.,.,.,.,.{GLASS\   .,..,.,.,.,.,..,.,.,.,.,.,.,..,.,.,.,
,..,.,,.,,.,{OF|MILK\..,.,.,.,.,..,.,.,.,.,.,..,.,.,.,.,.,.,
,.,..,.,,.,{ISWORTH},.,.,..,.,.,.,.,..,..,.,.,..,.,.,.,.,.,.
.,.,.,.,.{EVERYTNG}.-.-.--..-.-.-.-.--..--.-.-.-.-.--.-.-.-.
-.-.-.-{FORINFANTS}___--___-_-__-___--*(0~`~.,.,.,.,><><.><>
_-__-_{BUTBETTER}-.-,-,-,-,-,-,-,-,.-^^^^.-.-.-.-.^^^7>>>,..
.._...{WITH_HONEY}-.-.-.-.-.-.-.-.-.-.RANDOM(BUSH)SHRUBS>_..
GRASS_GRASS_GRASS_GRASS_GRASS_SOMEROCKS>GRASS>GRASS<GRASS>PC
SOIL_ROOTS_SOIL_SOIL_ROCKS_SOIL_GRASS_GRASS_GRASS_ROCKS_SOIL
CLAY_ROCKS_PEBBLES_CLAY_CLAY_CLAY_CLAY_GOLD_CLAY_CLAY><_WORM
ROOTS_CLAY_SKELETON_MORESOIL_CLAY_CLAY_CLAY_CLAY_<MUSHROOMS>
\end{verbatim}
\end{figure}

\begin{figure}
  \mn{alt=The Trojan Rabbit with an automatic sliding door, in ASCII}
  \label{trojan-rabbit}
  \caption{The Trojan Rabbit with an automatic sliding door}
\begin{verbatim}
                           ___  ____
                          //_ \//\__\
                            || ||  |
                         -__||_||__|
                       //         \--_
                      //     ____     --___
                     //     //   \         \-_
                    //      \\  @/        o ||
                   //        ----      _____||
                  //                   //
             //\_//__                 //
           //--  --- \____           //
          //          --- \______   //
         //   , .          ----- \_//_
        //       ,.               --- \____
       //              .,v             --- \___
      //                                 __ -- \_
     ||  ,         _______________       //||     |-_
     ||           |   |''''''''''|     // ||     |  |
     ||     '     |   |          |        ||     |  |
     ||           |   |          |        ||     |  |
     ||      "    |   | 0        |     ___||___  |  |
     ||           |   |          |     --------  |  |
     ||___        |   |          |        ______ |  |-
    //     \      |   |          |       //     \| _| \
   //       \ ____|---|__________|______//       \/    |
  ||    X    |      /                  ||    X    |   /
   \\       /\\____/                    \\       /___/
    \\_____/ -----                       \\_____/---
     -----                                -----
\end{verbatim}
\end{figure}

% [.comment]
% tag::aside1[]
% //tag::aside[]

% TODO PB: aside environment
\begin{aside}
  While the exchange at the French-occupied castle is one of
  the more memorable scenes of 
  \textbf{Monty Python and the Holy Grail},
  the Trojan Rabbit has not reached the same level of cultural
  resonance as its more murderous counterpart. Reasons for this
  may include:

  \begin{itemize}
    \item Less overall screen-time dedicated to the Trojan Rabbit.
    \item The Trojan Rabbit as projectile has already been anticipated
    by the Cow as projectile.
  \end{itemize}
\end{aside}

% //end::aside[]

% [.comment]
% end::aside1[]

% [.comment]
% tag::note1[]
% //tag::note[]

\begin{note}
  \mn{display=true,source=Author}
  Image courtesy of
  % TODO \href apparently not working
  {https://camelot.gov.example/creatures-in-ascii/}
\end{note}

% //end::note[]
% [.comment]
% end::note1[]


% [.comment]
% tag::comment1[]
% //tag::comment[]

The exchange of projectile animals was the beginning of a
long-running fruitful relationship between the British and the
French peoples, which
arguably predates the traditional English enmity with the
French.

% TODO: is this a comment?
\begin{comment}
  TODO: Will need to verify that claim.
\end{comment}

% TODO: is this a comment?
\begin{comment}
  Strictly speaking, the Knights are Welsh.
\end{comment}

% TODO: this is a comment, but how should it be formatted?
\begin{comment}
This document, as it turns out, has a profusion of XML comments.

As expected, they are ignored in any rendering of the document.
\end{comment}


% //end::comment[]
% [.comment]
% end::comment1[]

\subsection{The Mythos of Caerbannog}
\label{caerbannog}

% [.comment]
% tag::xref1[]
% //tag::xref[]

The \textit{Cave of Caerbannog} has been well-established in the mythology
of Camelot (as recounted by Monty Python) as the lair of the
Legendary Black Beast of Arrrghhh, more commonly known today as the
\textbf{Killer Rabbit of Caerbannog} \mncite{killer_rabbit_caerbannog}.
It is the encounter between the Killer Rabbit of Caerbannog and the
Knights of the Round Table, armed with the Holy Hand Grenade of
Antioch (see the \mncite{holy_hand_grenade,following section}), that we
recount here through monospace font and multiple spaces.

\subsubsection{The Killer Rabbit of Caerbannog}
\label{killer_rabbit_caerbannog}
% TODO: this label ain't behaving -- why?

% //end::xref[]
% [.comment]
% end::xref1[]

% [.comment]
% tag::relref1[]
% //tag::relref[]

The \textbf{Killer Rabbit of Caerbannog}, that most formidable foe of
the Knights and of all that is holy or carrot-like, has been
depicted diversely in lay and in song. We venture to say,
\textit{contra} the claim made in \mncite{RFC8140,4.1 of: Ze Vompyre},
that the Killer Rabbit of Caerbannog truly is the most afeared
of all the creatures. Short of sanctified ordnance such as
\mncite{holy_hand_grenade,format=title}, there are few remedies
known against its awful lapine powers.

% //end::relref[]
% [.comment]
% end::relref1[]

% [.comment]
% tag::hyperlink1[]
% //tag::hyperlink[]

\mncite{killer-bunny,The following depiction} of the fearsome beast
has been sourced from
\url{http://camelot.gov.example/avatars/rabbit[Rabbit-SCII]},
\mncite{killer-source,accompanied}
by C code that was used in this accurate depiction of the
Killer Rabbit:

% //end::hyperlink[]
% [.comment]
% end::hyperlink1[]

% [.comment]
% tag::figure1[]
% //tag::figure1a[]

\begin{figure}[h]\centering
  \label{killer-bunny}
  \caption{A Photo Of The Killer Rabbit of Caerbannog Taken In Secret}
  \alt{The Killer Bunny, in ASCII}
  \begin{verbatim}
....
\\\\\\\\\\\\\\\\\\\\\\\\\\\\\\\\\\\\\\\\\\\\\\\\\\\\\\\\\\\\
\\\\\\\\\\\\\\\\\\\\\\\\\\\\\\\\\\\\\\\\\\\\\\\\\\\\\\\\\\\\
\\\\\\\\\\\\\\\\\\\\\^^^^^^^^^^^^^^^^^^^^^^\\\\\\\\\\\\\\\\\
\\\\\\\\\\\\\\\\\\\<<#MWSHARPMWMWMWTEETHWMWWM>>>\\\\\\\\\\\\
\\\\\\\\\\\\\\\<<<#WMMWMWDEEPMDARKWCAVEMWWMMWM##>>>>\\\\\\\\
\\\\\\\\\\\\\<<#WMWMWMWMWWM/^MWMWMWMWMWMW^WMWMWMMW#>>>\\\\\\
\\\\\\\\\\\\<<#WMWMBEASTMW// \MWABBITWMW/ \MWMWMWMW##\\\\\\\
\\\\\\\\\\##MWMWMMWMWMWMWM\\  \MWMWMWMW/  /MWMWMWMWM##\\\\\\
\\\\\\\\##WMWMWMWMMWMWMWMWM\\  \MWMWMW/  /MWMWMWMMWMWMWM##\\
\\\\\\\##MWMMRAVENOUSMWMWMWM\\  \====/  /MWMRABBITMWMWMWMW##
\\\\\\##MWMWMWMWMMWMWMWMWMW[[            ]WMWMWMMWMWMWMWMWMW
\\\\\##MWMWMWMWCARNIVOROUSW[[   3    3   ]MWMWTOOMDARKWMWMMW
\\\\##MWMWDARKMWMWMWMWMWMWM//\     o    /MWMWMWMMWMWMWMMWMWM
\\##MWMWMMKILLERABBITWMWMM//| \___vv___/ \WMPITCHWBLACKWMWMW
\##MWMWMWMMWMWMWMWMWMMWMW// |   \-^^-/   |MWMWMWMMWMWMWMWMWM
MWMWMWMMWMWVERYMDARKWMMW//  |            |MWMCAERBANNOGWMWMW
MWMWMWMMWMWMWMWMWMWMWMM{{  /             /MWMWMMWMWMWMWMWMWM
MULTRADARKWMWMHELPMWMWMW\\ \  |      |  |MWMCANMMWMWMWMMWMWW
MWMWMWMWMMWMWMWMWMMWMWMWM\\ | |_     |  |_WMWMMYOUMWMMWWMWMW
MWMMWMWMWMWMBLACKWMWMWMWWM\_|__-\-----\__-\MWMWMWMREADMWMWWM
MWMWMWMMWMWMWMWMMWMWMWWMWMWMWMMWMWMWMWMWMWMWMWMWMWMWMMTHISWW
MWVERYMMSCARYMWMWWMWMMWMWMWMWMWMWMWMWMWMWMWMWWMWMMWMWIWM'.',
MWMWMMWMW======MWMMCANTWSEEMAMTHINGMMWMWMWMWMWMWMBETMMW` . `
MWMWMWM// SKULL \MWMWMWMMWSCREAMMMWMWMWMMWMNOTMWMWMWW  ` . \
MWMWMW|| |X||X| |MWMWCALLMMEWMMWMWMMWMWMWMWWM - ` ~ . , '
MWMWMW||___ O __|MWMWMWMMWMWMWMWMMW'   ___________//   -_^_-
MWMWMW \\||_|_||MWMW      '   . .     <_|_|_||_|__|     \O/
MW   \\/\||v v||  -\\-------___     .   .,         \     |
    \\|  \_CHIN/  ==-(|CARROT/)\>     \\/||//         v\/||/
       )          /--------^-^            ,.            \|//
 #  \(/ .\\|x//                              " ' '
  . ,                \\||//        \||\\\//   \\
\end{verbatim}

\end{figure}

% TODO: is this figure environment avoidable?
\begin{figure}[h]\centering
\label{killer-source}
\caption{C Code To Lure Killer Rabbit Back To Cave}
% TODO: \mn{source,c,markers=true}
\begin{sourcecode}
/* Locate the Killer Rabbit */
int type;
unsigned char *killerRabbit =
  LocateCreature(&caerbannog, "killer rabbit");
if( killerRabbit == 0 ){
  puts("The Killer Rabbit of Caerbannog is out of town.");
  return LOST_CREATURE;
}

/* Load Cave */
unsigned char *cave = LoadPlace(&caerbannog,
  "The Cave Of Caerbannog");
if( cave == 0 ){
  puts("The Cave of Caerbannog must have moved.");
  return LOST_PLACE;
}

/* Lure the Killer Rabbit back into the Cave */
unsigned char *carrot = allocateObjectInPlace(
  carrot("fresh"), cave);
if( carrot == 0 ){
  puts("No carrot, no rabbit.");
  return LOST_LURE;
}

/* Finally, notify the Killer Rabbit to act */
return notifyCreature(killerRabbit, &carrot);
\end{sourcecode}
\end{figure}


% //end::figure1a[]
% [.comment]
% end::figure1[]

On the beast's encounter with the Knights of the Round Table,
the following personnel engaged with it in combat:

% [.comment]
% tag::ul1[]
% //tag::ul[]

\begin{itemize}
  \item Killed
    \begin{itemize}
      \item Sir Bors
      \item Sir Gawain
      \item Sir Ector
    \end{itemize}
  \item Soiled Himself
    \begin{itemize}
      \item Sir Robin
    \end{itemize}
  \item Panicked
    \begin{itemize}
      \item King Arthur
    \end{itemize}
  \item Employed Ordnance
    \begin{itemize}
      \item The Lector
      \item Brother Maynard
    \end{itemize}
  \item Scoffed
    \begin{itemize}
      \item Tim the Enchanter
    \end{itemize}
\end{itemize}

% //end::ul[]
% [.comment]
% end::ul1[]


\subsubsection{Holy Hand Grenade of Antioch}
\label{holy_hand_grenade}

% [.comment]
% tag::figure2[]

% //tag::figure2a[]

% TODO: wrapping figure should not be rendered to ADOC for verbatim
\begin{figure}[h]\centering
\label{hand-grenade-figure}
\caption{The Holy Hand Grenade of Antioch (don't pull the pin)}
% [alt=Holy Hand Grenade of Antioch, in ASCII]
\begin{verbatim}
                        ______
                       \\/  \/
                      __\\  /__
                     ||  //\   |
                     ||__\\/ __|
                        ||  |    ,---,
                        ||  |====`\  |
                        ||  |    '---'
                      ,--'*`--,
                    _||#|***|#|
                 _,/.-'#|* *|#`-._
               ,,-'#####|   |#####`-.
             ,,'########|   |########`,
            //##########| o |##########\
           ||###########|   |###########|
          ||############| o |############|
          ||------------'   '------------|
          ||o  o  o  o  o   o  o  o  o  o|
           |-----------------------------|
           ||###########################|
            \\#########################/
             `..#####################,'
               ``..###############_,'
                  ``--.._____..--'
                     `''-----''`
\end{verbatim}
\end{figure}

% //end::figure2a[]

% [.comment]
% end::figure2[]


\begin{figure}[h]\centering
  \mn{foo}
  \label{sovereign-orb}
  \caption{The Sovereign's Orb made invisible}
  % TODO: \mn{link=https://upload.wikimedia.org/wikipedia/commons/b/bf/Coa_Illustration_Elements_Globus_cruciger.svg,align=right}
  % TODO: image::https://camelot.gov.example/sovereigns_orb.jpg[Orb,124,135]
  %       \includegraphics[width=135,height=124]{https://camelot.gov.example/sovereigns_orb.jpg}
  \includegraphics[width=\textwidth]{example-image-a}
\end{figure}

% [.comment]
% tag::index1[]
% //tag::index[]

% TODO: what are (((...))) ?
The solution to the impasse at the ((Cave of Caerbannog)) was
provided by the successful deployment of the
\textbf{Holy Hand Grenade of Antioch} (see \mncite{hand-grenade-figure})
(((Holy Hand Grenade of Antioch))).
Any similarity between the Holy Hand Grenade of Antioch and the
mythical \textit{Holy Spear of Antioch} is purely intentional;
(((relics, Christian))) any similarity between the Holy Hand Grenade
of Antioch and the \textit{Sovereign's Orb of the United Kingdom}
(see \mncite{sovereign-orb}) is putatively fortuitous.
(((relics, monarchic)))

% //end::index[]
% [.comment]
% end::index1[]

% [.comment]
% tag::dl1[]
% //tag::dl[]

\begin{description}
  \item[Holy Hand Grenade of Antioch]
    Ordnance deployed by Brother Maynard under the incantation of a
    lector, in order to dispense with the Foes of the Virtuous.
    See \mncite{hand-grenade-figure}.
  \item[Holy Spear of Antioch]
    A supposed relic of the crucifixion of Jesus Christ, this is one
    of at least four claimed instances of the lance that pierced
    Christ's side. Its historical significance lies in inspiring
    crusaders to continue their siege of Antioch in 1098.
  \item[Sovereign's Orb of the United Kingdom]
    Part of the Crown Jewels of the United Kingdom, the Sovereign's
    Orb is a hollow gold sphere set with jewels and topped with a
    cross.  It was made for Charles II in 1661. 
    See \mncite{sovereign-orb}.
\end{description}

% //end::dl[]
% [.comment]
% end::dl1[]

% [.comment]
% tag::bcp14_1[]
% //tag::bcp14[]

The instructions in the \textit{Book of Armaments} on the proper deployment
of the Holy Hand Grenade of Antioch \bcp{may} be summarized as
follows, although this summary \textbf{SHALL NOT} be used as a substitute
for a reading from the Book of Armaments:

% //end::bcp14[]
% [.comment]
% end::bcp14_1[]


% [.comment]
% tag::ol1[]
% //tag::ol[]

\begin{enumerate}
  \item Preamble: St Attila Benediction
  \item Feast of the People on Sundry Foods
    \begin{itemize}
      \item Lambs
      \item Sloths
      \item Carp
      \item Anchovies
      \item Orangutangs
      \item Breakfast Cereals
      \item Fruit Bats
      \item \textit{et hoc genus omne}
    \end{itemize}
  \item Take out the Holy Pin
  \item The Count
    \begin{enumerate}% TODO: [upperalpha]
      \item Count is to Three: no more, no less
      \item Not Four
      \item Nor Two, except if the count then proceeds to Three
      \item Five is Right Out
    \end{enumerate}
  \item Lob the Holy Hand Grenade of Antioch towards the Foe
  \item The Foe, being naughty in the \textbf{LORD's} sight, \bcp{shall} snuff it
\end{enumerate}

% //end::ol[]
% [.comment]
% end::ol1[]

This could also be represented in pseudocode as follows:

% [.comment]
% tag::listcontinuationblock1[]
% //tag::listcontinuationblock[]

\begin{enumerate}
  \item Take out the Holy Pin
  \item The Count
    \begin{verbatim}
    integer count;
    for count := 1 step 1 until 3 do
      say(count)
    comment Five is Right Out
    \end{verbatim}
  \item Lob the Holy Hand Grenade of Antioch towards the Foe
  \item Foe snuffs it
\end{enumerate}

% //end::listcontinuationblock[]
% [.comment]
% end::listcontinuationblock1[]

\subsection{Dramatis Personae}

The following human (more-or-less) protagonists were involved
in the two incidents recounted as lore of the Knights of the
Round Table:

% [.comment]
% tag::table1[]
% //tag::table[]


% TODO: table rendering is not advanced enough right now!
% TODO: [grid=all,options="footer"]
% \begin{longtabu} to \linewidth {|l|l|}
\begin{longtable}{|l|l|}

French Castle & Cave of Caerbannog \\
\endhead

Retinue of sundry knights &
Retinue of sundry more knights than at the French Castle \\
\endfoot

\endlastfoot

\multicolumn{2}{l}{King Arthur} \\
\multicolumn{2}{l}{Patsy} \\
\multicolumn{2}{l}{Sir Bedevere the Wise} \\
\multicolumn{2}{l}{Sir Galahad the Pure} \\
\multicolumn{2}{l}{Sir Lancelot the Brave} \\
\multicolumn{2}{l}{Sir Robin the Not-quite-so-brave-as-Sir-Lancelot} \\
French Guard with Outrageous Accent & Tim the Enchanter \\
Other French Guards & Brother Maynard \\
                    & The Lector \\
\multirow{3}{*}{not yet recruited} \\

\multicolumn{1}{r}{Sir Bors} \\
\multicolumn{1}{r}{Sir Gawain} \\
\multicolumn{1}{r}{Sir Ector} \\

\end{longtable}
% \end{longtabu}

% //end::table[]
% [.comment]
% end::table1[]

\subsubsection{Past the Killer Rabbit}

Once the Killer Rabbit of Caerbannog (\mncite{killer-bunny}) had been
dispatched, the Knights of the Round Table uncovered the last
words of Joseph of Arimathea, inscribed on the Cave of Caerbannog
in Aramaic.  While the precise Aramaic wording has not survived,
we trust the following Hebrew subtitles will serve as an
acceptable substitute:

% [.comment]
% tag::hebrew1[]
% //tag::hebrew[]

\begin{quote}
  \mn{Joseph of Arimathea, https://context.reverso.net}
% .כאן אולי ימצאו המילים האחרונות של יוסף מארמתיה
% .מי אשר יהיה אמיץ ובעל נפש טהורה יוכל למצוא את הגביע הקדוש בטירת אאאאאאאה

"Here may be found the last words of Joseph~of Arimathea.
He who is valiant and pure of spirit may find the Holy Grail
in the castle of --- Aaaargh."
\end{quote}

% //end::hebrew[]
% [.comment]
% end::hebrew1[]


\subsection{IANA Considerations}

IANA might consider a registry to track the mythical, especially
ravaging beasts, such as the Killer Rabbit, who haunt the Internet.


\subsection{Security Considerations}

Do not let the Killer Rabbit out under any circumstance.

I repeat. Do not let the Killer Rabbit (\mncite{killer-bunny}) out.


% [.comment]
% tag::bibliography1[]
% //tag::bibliography[]

% TODO: bibliography with custom title \subsection{Normative References}
\begin{thebibliography}{1}
  \bibitem[RFC 2119]{RFC2119}
\end{thebibliography}

% TODO: bibliography with custom title \subsection{Informative References}
\begin{thebibliography}{1}
  \bibitem[Grail]{grail_film}
    G. Chapman, J. Cleese, E. Idle, T. Gilliam, T. Jones, M. Palin. 1975. \textit{Monty Python and the Holy Grail}.
  \bibitem[RFC 2635]{RFC2635}
  \bibitem[RFC 7990]{RFC7990}
  \bibitem[RFC 8140]{RFC8140}
  \bibitem[RFC 8174]{RFC8174}
\end{thebibliography}

% //end::bibliography[]
% [.comment]
% end::bibliography1[]

\end{document}

